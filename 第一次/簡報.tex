\documentclass[12pt, svgnames]{article}
\usepackage{xeCJK}

\setCJKmainfont{Wt004.ttf}
\newfontfamily{\mytitlefont}{SavoyeLetPlain}

\usepackage{geometry}
\geometry{papersize = {128mm, 96mm},top = 0mm, bottom = 0mm, left = 0mm, right = 0mm}

\usepackage{amsmath}

\usepackage{tikz}
\usetikzlibrary{calc,arrows, mindmap}

\usepackage[minted]{tcolorbox}
\tcbuselibrary{raster}
\usepackage{pgffor}

\usepackage{multicol}

\usepackage{minted}

\usepackage{enumitem}

\definecolor{mygreen}{HTML}{608066}
\definecolor{mylightgreen}{HTML}{aad2ba}
\definecolor{mydarkgreen}{HTML}{1d1e18}

\pagecolor{mylightgreen}
\color{mydarkgreen}

\usepackage{graphicx}
\usepackage{animate}

\newtcolorbox{tinyinfo}{size = small, outer arc = 0mm, arc = 0mm, colback = mygreen, coltext = mydarkgreen, fontupper = \tiny, colframe = mygreen, }

\newtcolorbox{smallinfo}{size = small, outer arc = 0mm, arc = 0mm, colback = mygreen, coltext = mydarkgreen, fontupper = \small, colframe = mygreen, }

\newtcolorbox{info}{size = small, outer arc = 0mm, arc = 0mm, colback = mygreen, coltext = mydarkgreen, fontupper = \footnotesize, colframe = mygreen, halign = center}

\newtcolorbox{ofni}{size = small, outer arc = 0mm, arc = 0mm, colback = mydarkgreen, coltext = mygreen, fontupper = \footnotesize, colframe = mydarkgreen, halign = center}

\newtcolorbox{titlebox}{arc = 0mm, colback = mygreen, coltext = mydarkgreen, colframe = mygreen, width = 0.7\textwidth}

\tikzset{
	pics/mynode/.style args={#1}{
		code={
			\draw[fill=mygreen] (0,-15) -- (-75:9) arc (-75:255:9)--cycle;
			\node[circle, minimum size=1.2cm, fill=white, draw=black, label=center:{#1}] at (0,0) { };
     }
  }
}

\tikzset{
  	pics/upnode/.style args={#1}{
		code={
			\draw[fill=mygreen] (0,15) -- (105:9) arc (105:435:9)--cycle;
			\node[circle, minimum size=1.2cm, fill=white, draw=black, label=center:{#1}] at (0,0) { };
     }
  }
}

\tikzset{
	pics/file/.style args={#1#2}{
		code = {
		\begin{scope}[rotate = #2, scale = 2]
		\draw[fill = white] (-7, -10)--(-7, 10)--(4, 10)--(7, 7)--(7, -10)--cycle;
		\node[rotate = #2] at (0,0) {#1};
		\end{scope}
		}
	}
}

\title{第一次進度報告}
\author{第六組}

\makeatletter

\def\bottom@columns{5}
\def\first@topic{概覽}
\def\second@topic{$\phantom{\text{概}}$window}
\def\third@topic{$\phantom{\text{概}}$surface}
\def\fourth@topic{$\phantom{\text{概}}$event}
\def\fifth@topic{實戰}

\usepackage{scrextend}

\def\draw@top#1{%
\begin{tcbraster}[raster columns=1, raster equal skip= 0mm, raster column skip = 0pt, raster valign = bottom, raster after skip = 2.5mm]
\begin{tinyinfo}\centering\@title\end{tinyinfo}%
\begin{tcbraster}[raster columns=2, raster equal skip= 0mm, raster equal height, raster valign = top]%
\begin{smallinfo}$\phantom{\@author}$\hfill #1\end{smallinfo}%
\begin{smallinfo}\@author$\phantom{\text{#1}}$\end{smallinfo}%
\end{tcbraster}
\end{tcbraster}
}

\def\draw@bottom#1{
\begin{tcbraster}[raster columns=\bottom@columns, raster equal skip= 0mm, raster equal height]
\foreach \i in {\first@topic, \second@topic, \third@topic, \fourth@topic, \fifth@topic}
{\begin{\if\thetcbrasternum#1ofni\else info\fi}\i \end{\if\thetcbrasternum#1ofni\else info\fi}}
\end{tcbraster}}

\newenvironment{mainpage}[2]{
\def\current{#1}\draw@top{#2}\vskip12pt
\begin{addmargin}[10mm]{10mm}
}
{\end{addmargin}
\vskip12pt\vfill
\draw@bottom{\current}
}

\newenvironment{tikzpage}[2]{
\def\current{#1}\draw@top{#2}
}
{\vfill
\draw@bottom{\current}
}

\renewcommand{\maketitle}{
\mbox{ }\vfill%
{
\begin{center}
\begin{tcbraster}[raster columns = 1, raster equal skip = 0mm, raster width = 0.7\textwidth, raster halign = center]
\begin{tcolorbox}[outer arc = 0.5mm, arc = 0mm, colback = mygreen, coltext = mydarkgreen, fontupper = \huge\centering, colframe = mygreen]
{\@title}
\end{tcolorbox}
\begin{tcbraster}[raster columns = 2, raster equal skip = 0mm]
\begin{tcolorbox}[colback = mylightgreen, colframe = mylightgreen]

\end{tcolorbox}
\begin{tcolorbox}[outer arc = 0.5mm, arc = 0mm, colback = mygreen, coltext = mydarkgreen, colframe = mygreen]
{\mytitlefont\@author}
\end{tcolorbox}
\end{tcbraster}
\end{tcbraster}
\end{center}
}
\vfill
}

\newcommand{\mytitle}[1]{%
\mbox{}\vfill\centering%
\begin{titlebox}
\centering{\huge #1}
\end{titlebox}
\vfill\mbox{}}
\newcommand{\subtitle}[1]{%
\mbox{}\vfill\centering%
\begin{titlebox}
\centering{\huge #1}
\end{titlebox}
\vfill\mbox{}}

\newcommand{\mysubtitle}[1]{%
\mbox{}\vfill\centering%
\begin{tcolorbox}[arc = 0mm, colback = mylightgreen, coltext = mydarkgreen, colframe = mylightgreen, width = 0.7\textwidth]
\centering{\LARGE #1}
\end{tcolorbox}
\vfill\mbox{}}

\makeatother

\linespread{1.35}
\author{Si smanglam}
\title{Pygame 讀書會\\第一週}
\begin{document}
\maketitle
\newpage

\begin{mainpage}{0}{大綱}
\begin{itemize}
\item 概覽
\item pygame window
\item pygame surface
\item pygame event
\item 實戰
\end{itemize}
\end{mainpage}
\newpage

\mytitle{遊戲的組成?}
\newpage

\begin{tikzpage}{1}{遊戲的組成}
\begin{tikzpicture}[x = 1mm, y = 1mm]
\draw[draw = none] (-60, -35) rectangle (60, 35);
\node (event) at (30, -15) {事件處理};
\node (art) at (0, -15) {碰撞偵測};
\node (window) at (-30, -15) {畫面渲染};
\node (game) at (0, 15) {遊戲};

\draw (0, 5)--(30, 5)--(event.north);
\draw (0, 5)--(art.north);
\draw (0, 5)--(-30, 5)--(window.north);
\draw (game.south)--(0, 5);

\end{tikzpicture}
\end{tikzpage}
\newpage

\begin{tikzpage}{1}{遊戲的組成}
\begin{tikzpicture}[x = 1mm, y = 1mm]
\draw[draw = none] (-60, -35) rectangle (60, 35);
\node (event) at (30, -15) {事件處理};
\node (window) at (-30, -15) {畫面渲染};
\node (game) at (0, 15) {遊戲};

\draw (0, 5)--(30, 5)--(event.north);
\draw (0, 5)--(-30, 5)--(window.north);
\draw (game.south)--(0, 5);

\end{tikzpicture}
\end{tikzpage}
\newpage

\begin{mainpage}{1}{程式結構}
\vfill
\begin{tcolorbox}
\begin{minted}[tabsize = 4]{text}
初始化
全域變數
遊戲主迴圈
	事件處理
	畫面渲染
\end{minted}
\end{tcolorbox}
\end{mainpage}
\newpage

\mytitle{Window}
\newpage

\begin{mainpage}{2}{window}
\vfill
\begin{description}
\item Window:
\item[含義:] 螢幕上的遊戲視窗,只要是渲染到這上面的東西都會出現在畫面上
\item[創建:] \verb|pygame.display.set_mode((長, 寬), 額外特性)|
\item[限制:] 一次只能有一個與顯示器連在一起的 window
\item[小知識:] 他會 return 一個 surface,所以其實我們可以直接跳到下一章。
\end{description}
\end{mainpage}
\newpage

\begin{mainpage}{3}{surface}
\vfill
\begin{itemize}
\item Surface:
\begin{itemize}
\item Pygame 中類似圖片的概念
\item 是處理渲染最重要的東西
\item 只要沒有被渲染到主 surface 上都不會被顯示出來
\item 可以通過多種方式創建
\end{itemize}
\end{itemize}
\end{mainpage}
\newpage

\begin{mainpage}{3}{surface 與常用功能}
\vfill
\begin{itemize}
\item \verb|pygame.display.update()|
\item 變數名稱.blit(子 surface, (x, y))
\item 變數名稱.fill((R, G, B))
\item 變數名稱.fill(顏色名稱)
\item pygame.draw.圖形(surface, (x, y))
\item pygame.image.load(圖片路徑) -> surface
\end{itemize}
\end{mainpage}
\newpage

\begin{mainpage}{4}{Event}
\vfill
\begin{itemize}
\item Event
\begin{itemize}
\item 這是 Pygame 中的事件
\item 一般會以 \verb|pygame.event.get()| 去得到以 List 方式儲存的事件。
\item 事件可以用 .type 的方式去確認這是哪種事件
\item 事件有 .dict 與 .type 這兩個屬性去儲存詳細的資料
\end{itemize}
\end{itemize}
\end{mainpage}
\newpage

\begin{mainpage}{4}{常用的 Event.type}
\vfill
\begin{itemize}
\item KEYDOWN
\item MOUSEBUTTONDOWN
\item MOUSEMOTION
\end{itemize}
\end{mainpage}
\newpage

\begin{tikzpage}{4}{Event Loop 範例}
\begin{tcolorbox}
\small
\begin{minted}[]{python}
for event in pygame.event.get():
	if event.type == pygame.KEYDOWN:
		print("鍵盤輸入偵測")
		if event.key == pygame.K_a
			print("a 鍵被按下")
			
	elif event.type == pygama.MOUSEBUTTONDOWN:
		print("滑鼠被按下")
	
\end{minted}
\end{tcolorbox}
\end{tikzpage}
\newpage

\mysubtitle{\normalsize
\begin{enumerate}
\item 創建一個視窗
\item 將這個視窗渲染上色
\item 讓使用者點滑鼠視窗會變色
\item 放一個會跟著使用者鍵盤輸入移動的方塊
\item 放一個在滑鼠點擊時會移動到滑鼠點擊位置的方塊
\end{enumerate}
}
\end{document}